\documentclass{article}

\usepackage[utf8]{inputenc}

\title{Rapid shortening at the eastern margin of the Tibetan plateau prior to the 2008 Mw=7.9 Wenchuan earthquake}
\author{T. Ben Thompson, Brendan J. Meade}
\date{September 2014}
\usepackage{graphicx}

\begin{document}

\maketitle

\section{Abstract}
The Longmen Shan is the steepest topographic front of the India-Asia collision and was the site of the Mw=7.9 Wenchuan earthquake.
Shortening estimates across the Longmen Shan provide strain accumulation rates and clarify the eastward extrusion of the Tibetan plateau.
Here, to explain the interseismic GPS velocities across the greater Longmen Shan region, we develop a boundary element model including earthquake cycle effects, topography, the westward dipping Beichuan fault, and a ~20 km deep, shallowly dipping, detachment.
The detachment is inferred from observations of slip during and after the Wenchuan earthquake and from structural considerations.
Previous analyses which neglected the detachment and earthquake cycle effects have found shortening rates near zero.
In contrast, we find that interseismic GPS data are consistent with a shortening rate of 6$\pm$2 mm/yr.
These results suggest that the Longmen Shan is an active fold-and-thrust belt with Wenchuan style earthquake recurrence intervals of $<$600 years.

\section{Introduction}
The Longmen Shan range is located on the eastern margin of the Tibetan and rises 4000m over 30km (CHECK THESE NUMBERS ON GOOGLE EARTH).
The 2008 $\mathrm{M_w}$ 7.9 Wenchuan earthquake ruptured 250km along strike on the Beichuan and Pengguan faults, resulting in approximately XXXXX fatalities (CITE STONE)
Prior geodetic estimates suggest almost zero horizontal shortening (Chen 2000, Shen 2005, etc.), in contrast with the dominant thrust slip-sense during the Wenchuan rupture. 
The lack of shortening is reconciled with the steep topography by appealing to a theory of lower crustal inflation whereby high topography can be formed without upper crustal shortening (CITE clark, royden, england, molnar, kirby, etc)
However, the magnitude and location of the Wenchuan earthquake suggest a more typical fold-and-thrust belt.
Using seismic reflection data, Hubbard and Shaw find almost 100\% shortening in the range-front 
Following on these works, there is a growing body of literature supporting the interpretation of the Longmen Shan as a fold-and-thrust belt (CITE, CITE, CITE) as opposed to a product of viscous lower crustal inflation.

Consistent with the fold-and-thrust mechanism
However, it remains to be explained how slow present-day geodetic shortening estimates can be reconciled with the Wenchuan earthquake and the large structurally-observed shortening.
Here, we present a model that explains the regional geodetic velocities as the effect of a complex interseismically locked fault geometry, including a 20km deep detachment as well as the range-front thrust Beichuan fault.
This geometry is consistent with the structural interpretations of Hubbard,Shaw,Klinger, with coseismic and postseismic Wenchuan slip distributions that attribute significant motion to a deep detachment (CITE QI, FIELDING).  
Also, seismic observations of P/sP conversion in the crust are suggestive of a low velocity layer near 20km depth (CITE ZHANG ET. AL. CRUSTAL STRUCTURE). 
Using this range-front thrust plus detachment geometry, we infer an average horizontal shortening.
Next, we show how this geometry can drives the steep velocity gradients far to the northwest of the range front, thus explaining the absence of a range-front interseismic velocity gradient.
Finally, we discuss the effects of Longmen Shan fault geometry on regional seismic hazard.


The Longmen Shan are a mountain range at boundary between the Tibetan Plateau and the Sichuan Basin. West of the Longmen Shan, in the eastern Tibetan plateau, surface GPS velocity vectors rotate from northeasterly to easterly or south-easterly. 
This rotation has been attributed to either the extrusion of crustal blocks or topographically driven viscous flow impacting a cold, elastically and viscously rigid, Sichuan basin.
Because of its uniquely steep topographic gradient, lack of foreland basin, etc..., the Longmen Shan is a natural testing ground for these hypotheses.
The steep topographic front of the Longmen Shan has been explained by both horizontal shortening (cite meade, hubbard/shaw) and lower crustal inflation (cite clark, royden, england, molnar, others). 
The contrast between high topography and low shortening rates is a central point in this debate.
Cite kirby
The Mw 7.9 2008 Wenchuan earthquake ruptured 250km NE-SW along the Pengguan and Beichuan faults in the Longmenshan (cite xu 2008);e .
Average slip was 4m 
80,000 fatalities. How much slip.
Using InSAR measurements and post-Wenchuan GPS observations, Qi et. al. 2011 found evidence of 2-6m of slip on a 20km deep detachment that extends 60km northwest underneath the Tibetan Plateau. 
Our goal is to demonstrate that locking on this deep detachment is observable in the interseismic GPS field prior to the Wenchuan earthquake.
Further, we will show how a deep detachment in the Longmen Shan shifts the main velocity gradient far to the northwest of the Beichuan fault trace.
Hence, the detachment obscures the near-field geodetic shortening signature. 
Accounting for the detachment, we present increased estimates of the shortening rate in the Longmen Shan.
Fielding and .. (CITE) interpret high resolution satellite gravity observations to indicate that the Tibetan Plateau is being thrust over the Sichuan Basin.

\section{Geodetic analysis of Longmen Shan shortening}
Geodetic shortening across the Longmen Shan
Show figure with the location of the profile, location of the GPS stations, beichuan fault trace, DEM background.
GPS observations from Apel, Banerjee, Calais, Gan, Vigny, assembled into a single reference frame by Loveless and Meade.
All observations were made prior to the Wenchuan earthquake.
Although sparse, improvements in interseismic GPS resolution will not be available for a long time (many decades?) due to overprinting postseismic signals from Wenchuan.

The debate over shortening in the Longmen Shan has centered around the clear lack of velocity gradient at the range front.
Chen et. al. 2000 < 3mm/yr of shortening in the Longmen Shan.
However, all their stations are within 50km of the range front. CHECK THIS
Reference the plot of GPS velocities in the figure which shows the best fitting solution.
General theme of previous interpretations: either ignored earthquake cycle effects or only looked at the near-field GPS.
Which, if there is a deep detachment, are useless because they would indicate almost no shortening.
Observations may also have been diminished because they were late in the earthquake cycle – just before Wenchuan.
Interseismic locking leads to a smooth profile of geodetic velocities across a locked fault.
Interpreting this profile is highly dependent on fault geometry.
In the Longmen Shan, a deep detachment leads to a surface velocity gradient that is localized very far to the west of the range front.
However, ignoring the detachment would indicate minimal shortening across the range front (see figures).
We developed a 2D boundary element model.
We assume isotropic elasticity.
How much is worth saying about the BEM code?
The standard seems to be almost nothing...
I'm uncomfortable with that.
Should I link to a copy of the code used?

Our fault geometry is based on a community fault model of the Longmen Shan fault system (in development, shaw, plesch) that is based on seismic reflection imagery.
Although not directly observed, based on structural considerations, they include the deep detachment.
The location of this detachment is approximately the same as the Qi et. al. 2011 detachment. We use the SRTM30\_Plus global DEM(becker, sandwell et. al.)  to provide a representative, smoothed, topographic profile across the range. 
I need to talk to John Shaw to get reasonable elastic moduli for the region.
For topography, GPS, and fault geometry, we project the data onto the vertical plane of the cross section line (this should just go in a figure).
We fit the GPS velocities with a constant shortening rate on the whole Beichuan + detachment. The best fit is ~8 mm/yr (+-!!). 
I need to actually do some fitting to get uncertainties and best fit!

\section{Implications for tectonics and earthquake hazard}
A detachment underneath the Longmen Shan and Tibetan Plateau indicates that the Tibetan Plateau is thrust over the Sichuan Basin. I should read some of the previous literature on this question. Is there any evidence for Sichuan-like crust beneath easternmost Tibet? 
Hence, our detachment-based interpretation of the geodetic velocities supports a block-like interpretation for the tectonics of the eastern Tibetan Plateau. Lower crustal flow may also be able to explain the steep velocity gradient localized to the west of the range front. However, observations of postseismic afterslip on the detachment surface corroborate our  

Longriba fault zone?

I need to calculate recurrence intervals and look up previous estimates for comparison. 
Assuming 5m of slip in the Wenchuan earthquake, a 45 degree dipping fault, and 8mm of shortening: 8*sqrt(2) mm/yr slip deficit -> 5000/8*sqrt(2) = 441 years.
Assuming that all 8 mm/yr of slip-deficit is concentrated on a single 45 degree dipping fault with the same surface area as the Wenchuan earthquake rupture, we calculate a recurrence interval of 450 years.
However, this recurrence interval ignores the presence of other Longmen Shan faults and should be considered as an upper bound. The Beichuan and Pengguan faults are only two members of an imbricated thrust sequence. Interpretations of seismic reflection data (hubbard et. al) find multiple other thrusts including a detachment called the Range Front Thrust that extends 50km into the Sichuan Basin. This detachment lies directly under Chengdu, a city of 14 million. The slip-deficit distribution amongst these faults is unknown. Accounting for the subhorizontal Range Front Thrust, which has a large surface area, might seriously reduce our recurrence interval.

150kms west of the Beichuan, there are 3 GPS stations with similar profile location. But, the velocities are significantly different. This indicates some lateral heterogeneity in the orogeny.

The strain accumulating on the detachment may release coseismically or may be released in any number of slower ways. 
An analysis without the detachment would observe almost no horizontal velocity gradient across the Longmen Shan range front. This emphasizes that geometrically accurate fault models can dramatically alter the interpretation of geodetic data. 
Importance of geometry in seismic hazard. Even including earthquake cycle effects, we would estimate almost zero shortening without the presence of the detachment. Raise the impact of detachments on t

\subsection{A hazard lower bound}
There are reasons to believe that the seismic hazard estimate presented here is a low estimate.
First, coseismic and long term slip estimates include (CITE VARIOUS COSEISMIC + DENSMORE) a significant component of strike-slip motion. But, we only model horizontal shortening. 
Second, our two-dimensional model assumes plane strain conditions. 
The approximation will result in under-estimates of the shortening required to produce far-field velocities.
This is because the assumption is equivalent to assuming the fault geometry extends infinitely far along strike, resulting in a fault surface much larger than in reality. 
Third, the modeled range-front thrust is the steeply dipping Beichuan fault. The foreland Range Front thrust aor the shallow detachment beneath Chengdu (CITE HUBBARDSHAW) dip more shallowly, producing a greater total moment deficit.
These first three deficiencies could be remedied through a more geometrically realistic three-dimensional model of the region. However, the data density may not justify such a complex model. 
(MAYBE REMOVE) Also, kinematically consistent three-dimensional modelling of interseismic deformation is an outstanding issue. 
Finally, the shortening rate estimate may be low due to being estimated late in the earthquake cycle (GOOD CITATION FOR THE GENERAL CONCEPT?). 

On the other hand, there are reasons to believe that the strain energy budget in the Longmen Shan region is not entirely released in major earthquakes. Critical-taper wedge theory results in very low friction coefficients for the shallowly dipping detachment in the foreland and the deep detachment (Hubbard,Shaw, Klinger + deep detachment here). Hence, these fault segments are more likely to release significant energy aseismically. Analogously, it has been suggested that the Himalayan range front consumes most of its energy budget aseismically (CITE!!!).

\section{Conclusions}
Previous interseismic shortening estimates suggested almost no shortening. This is at odds with the Mw 7.9 Wenchuan earthquake. We clarify this conflict by including earthquake cycle effects and a 20km deep detachment. Our revised interseismic shortening rate is ().
This interpretation of the interseismic GPS velocities unifies the three phases of the Longmen Shan earthquake cycle. The Mw 7.9 Wenchuan rupture and deep postseismic afterslip suggest an active fold-and-thrust belt in the Longmen Shan orogen. Unlike previous interseismic coupling estimates (cite), our model is aligned with this view of the Longmen Shan. 
Our model suggest that Wenchuan-like events have much shorter recurrence intervals. Yikes!

\end{document}
